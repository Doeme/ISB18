\section{Results}\label{sec:results}
	At a first stage, yeast and lactic bacteria monocultures were simulated to verify whether they work independantly of each other.
	At the second stage, the coculture was simulated to show.
	
	The mediums used contain \SI{50}{\litre} of water mixed with an inital concentration of \SI{1000}{\milli\mole\per\litre}
	\subsection{Yeast Monoculture}
		The yeast monocolture is well-behaved, until the very edges of metabolic activity, where the GEM was blocked due to almost-zero uptake boundaries.
		But its growth is almost exponential, as can be seen in Figure \ref{fig:yeast_0lac_pop}.
		The steep curve at the beginning marks the region where oxygen can be metabolised.
		The slight curvature of the following section can be explained by the self-inhibition of the organism,
		caused by the toxicity of ethanol, and by the sinking glucose concentration.
		Concentrations of the relevant metabolites that can be seen in Figure \ref{fig:yeast_0lac_met}
		
		At time $t\approx 25$ the GEM is blocked since the glucose uptake can not satisfy the ATP maintenance reactions anymore;
		the population starts to die.
		\begin{figure}[h]
			\centering
			\includegraphics[width=\linewidth]{figures/results/yeast/0lac_populations.pdf}
			\caption{The yeast growth in the medium in a logarithmic scale}
			\label{fig:yeast_0lac_pop}
		\end{figure}
		\begin{figure}[h]
			\centering
			\includegraphics[width=\linewidth]{figures/results/yeast/0lac_metabolites.pdf}
			\caption{The concentrations of interesting metabolites in the simulated medium for yeast monocultures}
			\label{fig:yeast_0lac_met}
		\end{figure}
		
		In further simulations the concentration of lactate was increased, and the toxicity of the lactate was tuned to best fit the given curves.
		A number of $i_{Y,\mathrm{glc,lac}}=\SI{35}{\milli\mole\per\litre}$
		The Figure \ref{fig:yeast_simfig} shows the the fit to the curve given in $\ref{fig:yeast_real}$
		
		The fit is not perfect on multiple accounts: Firstly, the organism takes longer to metabolise the glucose, which can be traced back to a too high ethanol inhibition constant $i_{Y,\mathrm{glc,etoh}}$
		Secondly, there are sharp, seemingly non-smooth corners in the plot. These can be explained by wrong Michaelis-Menten constants $k_{Y,m}$
		
		Adapting these constants to $k_{Y,\mathrm{glc}}=\SI{15}{\milli\mole\per\litre}$ and the inhibition constant to $i_{Y,\mathrm{glc,etoh}}=\SI{350}{\gram\per\liter}$ yields a better fitting result
		shown in Figure \ref{fig:yeast_better_simfig}
		
		\begin{figure}[h]
			\centering
			\includegraphics[width=0.75\linewidth]{figures/yeast_real.png}
			\caption{Glucose and ethanol concentrations measured in a real-life scenario with varying lactate concentrations \cite{Narendranath2001}. }
			\label{fig:yeast_real}
		\end{figure}
		
		\begin{figure}[h]
			\centering
			\includegraphics[width=\linewidth]{figures/results/yeast/similar_plot.pdf}
			\caption{The ethanol and glucose concentration at different concentrations of lactate}
			\label{fig:yeast_simfig}
		\end{figure}

		\begin{figure}[h]
			\centering
			\includegraphics[width=\linewidth]{figures/results/better_yeast/similar_plot.pdf}
			\caption{The ethanol and glucose concentration at different concentrations of lactate with the improved constants}
			\label{fig:yeast_better_simfig}
		\end{figure}
