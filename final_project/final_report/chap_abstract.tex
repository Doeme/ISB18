Bacterial contamination in alcoholic fermentation lowers the productivity of yeast and has a negative
influence on process costs and product quality. Finding new ways to enhance the process is neccessary.
Simulation methods using genome-scale models has gained popularity as they provide deep insight in
internal cell processes and are much cheaper than using real bacteria cultures. Latest implementations
uses dynamic flux balance analyses (DFBA) methods to simulate competitive co-cultures.
One successfully applied framework for Matlab is \textit{Dynamic Multispecies Metabolic Modeling}
(DMMM). In this work the implementation of the DMMM is discussed and implemented in python using
the COBRApy package. As the contamination of Lactobacillus plantarum is a major issue in alcoholic
fermentation in food industry, the simulator is demonstrated in a co-simulation of L. plantarum
and Saccharomyces cerevisiae. The results show...

