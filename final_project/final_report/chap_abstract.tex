Bacterial contamination in alcoholic fermentation affects the performance of yeast, can lower the
alcohol production by up to 30\% and has a negative effects on product quality.
Developing new ways to enhance process control is essential to compete with other suppliers on the
market but experimental techniques are very costly.
Simulation methods as dynamic flux balance analyses (DFBA) using genome-scale models (GEM) has
proven their capabilities in predicting bacteria growth and metabolic behavior in batch processes
and competitive bacteria cultures. \textit{Dynamic Multispecies Metabolic Modeling} (DMMM)
is a successful application of DFBA in a simulation framework using Matlab. In this work the 
implementation of the DMMM is analysed and implemented in python using the COBRApy package.
As the contamination of Lactobacillus plantarum is a major issue in alcoholic
fermentation in food industry, the simulator is demonstrated in a co-simulation of L. plantarum
and S. cerevisiae. The results verify proper performance of the simulator and confirm important
rule of thumb measures usually employed by brewers like aeration prior to fermentation.
