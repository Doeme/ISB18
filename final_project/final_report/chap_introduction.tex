\section{Introduction}\label{sec:introduction}

% "The lactic acid bacteria and wild yeasts are the main contaminants, affecting yeasts’ performance"
% "affect the performance and preservation of yeast's fermentation parameters, which may influence process parameters and yield"
% "competition between yeasts and microbial contaminants"
% "Lactobacillus bacteria are important in inducing yeast flocculation"
% "flocculation, reduction of yeast cells viability and decrease in process yield"
% "Yeast flocculation can decrease fermentation rate"
% " reduction in process yield can reach 20–30%"
% " necessity to use acids, antibiotics and antifoam agents [32,33]. This in turn culminate with the increase in production costs and changes in quality of the final product"

Alcoholic fermentation is applied in industrial scale in different markets like food industry
e.g. for wine, beer or whisky production but also other market segments like bio fuel production.
Reducing production costs while increasing the process efficiency and product quality
is a major goal in industry. In the fermentation process contamination from tanks, pipes,
centrifuges, heat exchangers or ingredients can have a great impact on process parameters
and must be controlled. Contamination can lead to a loss in process productivity of up to
30\%. The main contaminants are lactic acid bacteria and wild yeasts, lowering the performance
of the applied yeast. Lactobacillus bacteria contribute to induce yeast flocculation what in turn
reduces the viability of the cells, the fermentation rate and finally the process yield. Competition
between yeasts and microbial contaminants affects the process additionally
\cite{brexo_impact_2017}\cite{narendranath_effects_1997}.

In beer, wine or whisky production additional product quality aspects like color, viscosity
and flavor plays a big role and has a great influence on its price. On the other hand the process
itself underlies strict laws to guarantee transparency for the customers\cite{biergesetz_1993}
\cite{weingesetz_1994}. Additionally the growing organic food market calls for products
with even more strict processing standards\cite{demeter_processing_standards_2017}.

Chemicals like acids, antibiotics and antifoam agents as they are used in other industries
can not be applied here. These constraints increases the effort of finding optimal ways for
controlling product quality dramatically. As experimental development in laboratories is very
costly and not affordable by smaller breweries, simulation methods are gaining popularity.

% You could add here some examples to show the variety of modelling approches in zomorrodi's paper

A well established group of methods uses Genome-Scale Models (GEM) to model metabolic fluxes in 
bacteria \cite{zomorrodi_synthetic_2016}.
GEMs are created by analysing the information in the genome of a bacteria about its enzymatical
processes and extrapolating a network of metabolites and metabolite reactions which represents
the internal fluxes of this bacteria. Based on this model possible flux distributions can be
predicted using linear optimization methods, known as flux balance analysis (FBA)\cite{orth_what_2010}.

In its basic form describes FBA the bahavior of only one bacteria in a constant environment.
Zomorri et al. summarizes some approaches to extend FBA to multispecies simulations
(\textit{steady-state models}) in time variant environments (\textit{dynamic models})
and even considering spacial variations (\textit{spatio-temporal models})\cite{zomorrodi_synthetic_2016}.

The simulation of co-cultures using GEMs is still an emerging field. Every approach has
its strengths and weaknesses and must be chosen dependent on the application.
In \textit{Steady-state models} like compartmentalized community-level metabolic modeling a
common objective function for all involved species must be formulated. This approach has 
a relatively low computational effort but shows drawbacks in competitive co-cultures where
a common objective can not easily be formulated.
The spacial dimension of \textit{Spatio-temporal models} results in very high computational
effort and it has to be evaluated if this enhances the accuracy of the estimated behavior
in that case.

The class of \textit{dynamic models} described by Zomorrodi et al. uses a system of ordinary
differential equations (ODE)
to extend the static simulation approach of FBA with external system dynamics like the exchange
of metabolites and bacteria growth and is called dynamic flux balance analysis (DFBA)\cite{zomorrodi_synthetic_2016}.

Mehadevan et al. describes two basic categories of DFBA approaches: \textit{dynamic optimization
approach} (DOA) and \textit{static optimization approach} (SOA)\cite{mahadevan_dynamic_2002}.

In DOA the linear programming (LP) problem to predict the behavior of the bacteria and the
differential equations are reformulized to a non-linear programming problem (NLP). 
This approach has a very high computational effort\cite{hoffner_reliable_2013} compared to SOA
and has only been used for relatively small GEMs with up to 13 modeled fluxes and 8 metabolites
\cite{luo_dynamic_2006} \cite{luo_photosynthetic_2009}.

In SOA the simulation time is discretised by a defined time interval and a FBA is executed for
each time point. The results are used to iteratively solve the discretised differential equations
to update the environmental conditions at each time point. The described integration method in SOA
is similar to Euler-Cauchy.
In this approach it is assumed that the cell internal dynamics are much faster than the external dynamics.

Höffner et al. basically generalizes SOA to arbitrary integration methods for solving the system of ODEs
by using a third-party solver and names it \textit{direct approach} (DA)\cite{hoffner_reliable_2013}.

Henson et al. introduces a further method which shows similarities to SOA and DA but enhances the efficiency of the
algorithm\cite{hoffner_reliable_2013} by reformulating the LP and ODE to a 
differential-algebraic equation system\cite{henson_dynamic_2014}.

The DA described by Höffner et al. has been successfully applied to simulate alcohol fermentation
in a wine approximating setup \cite{sainz_modeling_2003} \cite{pizarro_coupling_2007} and to
predict growth of Escherichia coli and Saccharomyces cerevisiae bacteria in a co-culture
\cite{hanly_dynamic_2011}. Zhuang et al. implemented
the method in the \textit{Dynamic Multispecies Metabolic Modeling} (DMMM) framework
which is publicly accessible and written for Matlab\cite{zhuang_design_2012}. It uses the Matlab
toolbox \textit{COBRA}\cite{heirendt_creation_nodate} to implement FBA.
The method developed by Henson et al. compared to DMMM has the potential to increase the accuracy
by fixed calculation runtime but since the implementation of DMMM is publicly accessible and
relatively simple to implement, this method is used in the simulator in this work.

\update{
The next chapter will give a overview over the applied models and algorithms followed by the
description of a simulation setup including Saccharomyces cerevisiae and Lactobacillus plantarum
which shall be serve as a use-case. As a first step Sacharomyces cerevisiae is simulated with
different densities of lactic acid to tune the inhibition of the yeast model by the acid. In a
second step S. cerevisiae and L. plantarum are simulated in a co-culture.
The results are compared to the outcomes of \textit{in vivo} experiments and discussed in chapter \ref{sec:results}.
Chapter \ref{sec:conclusions} gives an outlook on further improvements of the implementation.}

 


% list a few papers where this approach was already successfully used

% argument that it's also easy to implement (see DMMM)


% of these time intervals dependent on the metabolite densities. The solution of the LP defines the bacteria growth and metabolite
% production at a certain point of time in the simulation time interval. These values are then used to solve the differential equations
% which models the external system dynamics. To solve the LP for the next time interval the new calculated, changed metabolite densities
% are used. This procedure is repeated until the end of the simulation time interval is reached. This approach makes use of the
% assumption that the cell internal dynamics are much faster than the external dynamics. In SOA the behavior of the bacteria is assumed
% to be constant during one time interval what leads to a linear approximation approach when solving the system of ordinary differential
% equations (ODE), similar to Euler-Cauchy methods.
% 
% Höffner et al. adds in \cite{hoffner_reliable_2013} a further group, the \textit{direct approach} (DA) which basically describes methods
% similar to SOA which uses an ODE solver instead of the Euler-Cauchy method. Due to the used ODE solver different numerical approximation
% methods can be used, not only the linear approximation. A good documented example for this group is the \textit{Dynamic Multispecies
% Metabolic Modeling} framework by Zhuang et al. \cite{zhuang_design_2012}.
% 
% Henson et al. mentions a third group, \textit{reformulation to a differential-glgebraic equation system} \cite{henson_dynamic_2014}.
% It shows also many similarities to SOA with the difference that the LP is reformulized but still solved as a LP embedded within 
% the external ODE. The reformulated equation system makes it possible to enhance the efficiency of algorithm compared to SOA and DA
% \cite{hoffner_reliable_2013}.

% FBA describes the metablic flux within a bacteria in a constant environment. To predict the
% behavior of the bacteria in a changing environment FBA was extended to dynamic flux balance
% analyses (DFBA). 
% 
% 
% 
% Traditional beer brewing is done in a batch process by fermenting glucose to alcohol using
% Saccharomyces cerevisiae. In a first step wort is produced by mixing water, malt and hop and
% applying different enzymatical processes. As in a batch process the densities of metabolites
% in the culture are not controlled the fermentation product and so the quality of the beer is
% highly dependent on the composition of the wort. To enhance the product quality the process
% of fermentation must be understood in detail and correlations between the starting conditions
% and fermentation results must be found. This is typically done in experiments which are very
% time and cost intensive. 
% 
% Especially high effort is needed to reproduce starting conditions and if
% different yeast mutants or contamination by other bacteria shall be tested. These experiments
% are very costly and are not affordable for smaller breweries. A simulation approach to test
% different starting conditions will reduce the amount of experiments and so the costs and will
% enable development of new production methods also for smaller companies.
% 
% As the formulation of sufficient models for (1) and (2)
% depends on the production process, so the applied yeast and wort, this project will concentrate
% on the development of a simulation framework to enable the simulation of the fermentation
% products dependent on the fermentation’s starting conditions.
% 
% 
% 
% 
% Mehadevan et al. introduces two basic categories of DFBA approaches: \textit{dynamic optimization approach} (DOA) and \textit{static
% optimization approach} (SOA)\cite{mahadevan_dynamic_2002}. In DOA a the linear programming problem (LP) which predicts the bacteria
% behavior is reformulized to a non-linear programming problem (NLP). This approach has a very high computational effort
% \cite{hoffner_reliable_2013} compared to SOA and has only been used for relatively small GEMs with up to 13 modeled fluxes and 8 metabolites
% \cite{luo_dynamic_2006} \cite{luo_photosynthetic_2009}.
% 
% Mehadevan et al. introduces SOA in \cite{mahadevan_dynamic_2002} as follows: The simulation interval is divided into several intervals and the LP is solved for each
% of these time intervals dependent on the metabolite densities. The solution of the LP defines the bacteria growth and metabolite
% production at a certain point of time in the simulation time interval. These values are then used to solve the differential equations
% which models the external system dynamics. To solve the LP for the next time interval the new calculated, changed metabolite densities
% are used. This procedure is repeated until the end of the simulation time interval is reached. This approach makes use of the
% assumption that the cell internal dynamics are much faster than the external dynamics. In SOA the behavior of the bacteria is assumed
% to be constant during one time interval what leads to a linear approximation approach when solving the system of ordinary differential
% equations (ODE), similar to Euler-Cauchy methods.
% 
% Höffner et al. adds in \cite{hoffner_reliable_2013} a further group, the \textit{direct approach} (DA) which basically describes methods
% similar to SOA which uses an ODE solver instead of the Euler-Cauchy method. Due to the used ODE solver different numerical approximation
% methods can be used, not only the linear approximation. A good documented example for this group is the \textit{Dynamic Multispecies
% Metabolic Modeling} framework by Zhuang et al. \cite{zhuang_design_2012}.
% 
% Henson et al. mentions a third group, \textit{reformulation to a differential-glgebraic equation system} \cite{henson_dynamic_2014}.
% It shows also many similarities to SOA with the difference that the LP is reformulized but still solved as a LP embedded within 
% the external ODE. The reformulated equation system makes it possible to enhance the efficiency of algorithm compared to SOA and DA
% \cite{hoffner_reliable_2013}.
% 
% The described DFBA methods in section \ref{ssec:considered_dfba_approaches} were rated based on the given information in the above mentioned
% papers, see table \ref{tab:rating_of_DFBA_methods}.
% 
% DOA can not be used due to its high computational effort and medium-high implementation complexity. The approach which uses
% \textit{reformulation to a differential-glgebraic equation system} is currently available in matlab code and must be implemented
% in python in this project. Due to the high implementation complexity this approach will also be excluded.
% The remaining methods, SOA and DA, have similar ratings but as DA is more flexible as different ODE solvers can be used this approach
% seems more sustainable. Besides its flexibility the DA implementation DMMM by Zhuang et al. \cite{zhuang_design_2012} can be publicly
% accessed and they provide a good documentation which will facilitate the implementation in this project.

