\section{Introduction}\label{sec:introduction}

Traditional beer brewing is done in a batch process by fermenting glucose to alcohol using
Saccharomyces cerevisiae. In a first step wort is produced by mixing water, malt and hop and
applying different enzymatical processes. As in a batch process the densities of metabolites
in the culture are not controlled the fermentation product and so the quality of the beer is
highly dependent on the composition of the wort. To enhance the product quality the process
of fermentation must be understood in detail and correlations between the starting conditions
and fermentation results must be found. This is typically done in experiments which are very
time and cost intensive. Especially high effort is needed to reproduce starting conditions and if
different yeast mutants or contamination by other bacteria shall be tested. These experiments
are very costly and are not affordable for smaller breweries. A simulation approach to test
different starting conditions will reduce the amount of experiments and so the costs and will
enable development of new production methods also for smaller companies.

As the formulation of sufficient models for (1) and (2)
depends on the production process, so the applied yeast and wort, this project will concentrate
on the development of a simulation framework to enable the simulation of the fermentation
products dependent on the fermentation’s starting conditions.
