\section{Introduction}\label{sec:introduction}

Traditional beer brewing is done in a batch process by fermenting glucose to alcohol using
Saccharomyces cerevisiae. In a first step wort is produced by mixing water, malt and hop and
applying different enzymatical processes. As in a batch process the densities of metabolites
in the culture are not controlled the fermentation product and so the quality of the beer is
highly dependent on the composition of the wort. To enhance the product quality the process
of fermentation must be understood in detail and correlations between the starting conditions
and fermentation results must be found. This is typically done in experiments which are very
time and cost intensive. Especially high effort is needed to reproduce starting conditions and if
different yeast mutants or contamination by other bacteria shall be tested. These experiments
are very costly and are not affordable for smaller breweries. A simulation approach to test
different starting conditions will reduce the amount of experiments and so the costs and will
enable development of new production methods also for smaller companies.

As the formulation of sufficient models for (1) and (2)
depends on the production process, so the applied yeast and wort, this project will concentrate
on the development of a simulation framework to enable the simulation of the fermentation
products dependent on the fermentation’s starting conditions.


Zomorri et al. summarizes in \cite{zomorrodi_synthetic_2016} models to predict the behavior of bacteria cultures and introduces
different categories. Three of them are especially
interesting to be used in this project: \textit{steady-state models}, \textit{spatio-temporal models} and \textit{dynamic models}.
\textit{Steady-state models} like compartmentalized community-level metabolic modeling can not be used
since a common objective can not be generally assumed, as it would be in a purely competitive co-cultures
\textit{Spatio-temporal models} have a very high computational effort as they take spacial and temporal varying bacteria densities
into account. As the spacial aspect is not necessarily required in this project a more optimal approach shall be preferred.
The remaining category of \textit{dynamic models} is a well established method to simulate microbial co-cultures in batch processes
and summarizes different extensions to dynamic flux balance analyses methods (DFBA)\cite{zomorrodi_synthetic_2016}. They use genome-
scale models (GEM) to simulate the behavior of the bacteria cultures and add differential equations to model the external system dynamics.

Mehadevan et al. introduces two basic categories of DFBA approaches: \textit{dynamic optimization approach} (DOA) and \textit{static
optimization approach} (SOA)\cite{mahadevan_dynamic_2002}. In DOA a the linear programming problem (LP) which predicts the bacteria
behavior is reformulized to a non-linear programming problem (NLP). This approach has a very high computational effort
\cite{hoffner_reliable_2013} compared to SOA and has only been used for relatively small GEMs with up to 13 modeled fluxes and 8 metabolites
\cite{luo_dynamic_2006} \cite{luo_photosynthetic_2009}.

Mehadevan et al. introduces SOA in \cite{mahadevan_dynamic_2002} as follows: The simulation interval is divided into several intervals and the LP is solved for each
of these time intervals dependent on the metabolite densities. The solution of the LP defines the bacteria growth and metabolite
production at a certain point of time in the simulation time interval. These values are then used to solve the differential equations
which models the external system dynamics. To solve the LP for the next time interval the new calculated, changed metabolite densities
are used. This procedure is repeated until the end of the simulation time interval is reached. This approach makes use of the
assumption that the cell internal dynamics are much faster than the external dynamics. In SOA the behavior of the bacteria is assumed
to be constant during one time interval what leads to a linear approximation approach when solving the system of ordinary differential
equations (ODE), similar to Euler-Cauchy methods.

Höffner et al. adds in \cite{hoffner_reliable_2013} a further group, the \textit{direct approach} (DA) which basically describes methods
similar to SOA which uses an ODE solver instead of the Euler-Cauchy method. Due to the used ODE solver different numerical approximation
methods can be used, not only the linear approximation. A good documented example for this group is the \textit{Dynamic Multispecies
Metabolic Modeling} framework by Zhuang et al. \cite{zhuang_design_2012}.

Henson et al. mentions a third group, \textit{reformulation to a differential-glgebraic equation system} \cite{henson_dynamic_2014}.
It shows also many similarities to SOA with the difference that the LP is reformulized but still solved as a LP embedded within 
the external ODE. The reformulated equation system makes it possible to enhance the efficiency of algorithm compared to SOA and DA
\cite{hoffner_reliable_2013}.

The described DFBA methods in section \ref{ssec:considered_dfba_approaches} were rated based on the given information in the above mentioned
papers, see table \ref{tab:rating_of_DFBA_methods}.

DOA can not be used due to its high computational effort and medium-high implementation complexity. The approach which uses
\textit{reformulation to a differential-glgebraic equation system} is currently available in matlab code and must be implemented
in python in this project. Due to the high implementation complexity this approach will also be excluded.
The remaining methods, SOA and DA, have similar ratings but as DA is more flexible as different ODE solvers can be used this approach
seems more sustainable. Besides its flexibility the DA implementation DMMM by Zhuang et al. \cite{zhuang_design_2012} can be publicly
accessed and they provide a good documentation which will facilitate the implementation in this project.

