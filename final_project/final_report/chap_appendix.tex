\appendices
\section{}\label{ap:super_fancy_stuff}
The following approximation is used to convert \textdegree P (``degree plato'') to a density measure (\si{\gram\per\liter})\cite{bubnik1995sugar}.
\begin{equation} \label{eq:grad_plato_to_density}
 d_{total} = 4.13 \frac{g}{l \,\,\, ^\circ P} p + 997 \frac{g}{l}
\end{equation}

As the simulation framework expects metabolite densities relative to the total volume of the solution (mmol of metabolite per liter
solution, mmol/l) the total density $d_{total}$ must to converted to a density $s_{glc}$.
It is assumed that $V_{total} = V_{glc} + V_W$. Furthermore $m$ is used in the following equations as a measure of mass.

\begin{align}\label{eq:total_density_to_metabolite_density}
 d_{total} & = \frac{m_{total}}{V_{total}} \nonumber \\
           & = \frac{m_{glc} + m_w}{V_{total}} \nonumber \\
           & = \frac{m_{glc} + d_w V_w}{V_{total}} \nonumber \\
           & = \frac{m_{glc} + d_w \left(V_{total} - V_{glc}\right)}{V_{total}} \nonumber \\
           & = \frac{m_{glc} + d_w \left(V_{total} - \frac{m_{glc}}{d_{glc}}\right)}{V_{total}} \nonumber \\
 \Leftrightarrow \quad s_{glc}   & = \frac{m_{glc}}{V_{total}} = \frac{d_{total} - d_w}{1 - \frac{d_w}{d_{glc}}}
\end{align}

Combining equation \ref{eq:grad_plato_to_density} and \ref{eq:total_density_to_metabolite_density}, including all constants and converting it to mmol/l leads to:
\begin{equation} \label{eq:ready_to_use_plato_to_metabolite_density}
 s_{glc} = \left( 63.857 \frac{1}{^\circ P} p - 46.385 \right) \frac{mmol}{l}
\end{equation}




\begin{table}[H]
\centering
\caption{Constants used in this document}
\label{tab:constants_used_in_this_document}
\begin{tabular}{llllll}
\rowcolor[HTML]{EFEFEF} 
\cellcolor[HTML]{EFEFEF} Constant    & \cellcolor[HTML]{EFEFEF}value & \cellcolor[HTML]{EFEFEF}reference\\
Oxygen saturation of water at 20\textdegree C (\si{\milli\gram\per\liter}) & 9.077 & \cite{fao_water_1987} \\
Molar mass of water (\si{\gram\per\mole})       &  18.015 & \cite{pupchen_website}\\
Molar mass of glucose (\si{\gram\per\mole})     & 180.156 & \cite{pupchen_website}\\
Density of water (\si{\gram\per\liter})         &    1.00 & \cite{pupchen_website}\\
Density of glucose (\si{\gram\per\liter})       &    1.56 & \cite{pupchen_website}\\
Considered wort densities (\textdegree P)       & 5...20 & - \\
\end{tabular}
\end{table}

To calculate the initial oxygen density in the solution it is assumed that the solution is at 20 \textdegree C and fully saturated
with oxygen:
\begin{align} \label{eq:init_oxygen_density}
s_{o} & = 9.077 \frac{mg}{l} \nonumber\\
            & = 9.077 \,\, 10^{-3}  \frac{g}{l} \nonumber\\
            & = \frac{9.077 \,\, 10^{-3} \frac{g}{l}}{18.015 \,\, 10^{-3} \frac{g}{mmol}} \nonumber\\
            & = 0,5039 \frac{mmol}{l} 
\end{align}

% you can choose not to have a title for an appendix
% if you want by leaving the argument blank
\section{}\label{ap:more_super_fancy_stuff}
\begin{table}[H]
\centering
\caption{Rating of considered DFBA methods\\ (1: low, 2: medium, 3: high)}
\label{tab:rating_of_DFBA_methods}
\begin{tabular}{llll}
\rowcolor[HTML]{EFEFEF} 
Method                                                                                                 & \begin{tabular}[c]{@{}l@{}}comp.\\ effort\end{tabular} & \begin{tabular}[c]{@{}l@{}}impl.\\ complexity\end{tabular} & flexibility \\
dynamic optimization approach (DOA)                                                                    & 3                                                   & 2-3                                                & ?           \\
static optimization approach (SOA)                                                                     & 1                                                    & 1                                                        & 1         \\
direct approach (DA)                                                                                   & 2                                                 & 1                                                        & 2      \\
\begin{tabular}[c]{@{}l@{}}reformulation to a differential-glgebraic\\ equation system\end{tabular}    & 1-2                                             & 3                                                       & ?          
\end{tabular}
\end{table}

\section{}
\begin{table*}[H]
\centering
\caption{Variables, constants and their units}
\label{tab:units_of_variables_and_constants}
\begin{tabular}{lll}
\rowcolor[HTML]{EFEFEF} 
Symbol                 & Unit    & Description\\
$M$                    & -       & bacteria model\\
$m$                    & -       & metabolite in shared medium\\
$X_M$                  & \si{\gram_{DW}} & dry weight of bacteria $M$ in shared medium\\
$x_M$                  & \si{\gram\per\liter} & density of bacteria $M$ in shared medium\\
$v_{M,m}$              & \si{\milli\mole\per\gram_{DW}\per\hour} & input/output flux of model $M$\\
$\mu_M$                & \si{\gram\per\gram_{DW}\per\hour} & growth rate of bacteria $M$\\
$S_m$                  & \si{\milli\mole} & amount of molecules of metabolite $m$ in shared medium\\
$s_m$                  & \si{\milli\mole\per\liter} & molecular density of metabolite $m$ in shared medium\\
$b_{M,m}$              & \si{\milli\mole\per\gram_{DW}\per\hour} & upper bound of input flux for metabolite $m$\\
$v_{max,M,m}$          & \si{\milli\mole\per\gram_{DW}\per\hour} & maximum input flux of metabolite $m$ at bacteria $M$\\
$k_{M,m}$              & \si{\milli\mole\per\liter} & Michaelis-Menten constant for metabolite $m$ at bacteria $M$\\
$I_{M,m,m'}$           & \si{\milli\mole\per\liter} & inhibition constant of bacteria $M$\\
$V$                    & \si{\liter} & total batch volume\\
 &&\\

\end{tabular}
\end{table*}

% \footnote{$\si{\gram_{DW}}$ stands for grams dry weight, i.e. grams of bacteria after all water has evaporated}
