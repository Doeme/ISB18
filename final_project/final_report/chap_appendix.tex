\appendices
\section{}\label{ap:super_fancy_stuff}
The following approximation is used to convert \textdegree C (``degree plato'') to a density measure (g/l)\cite{bubnik1995sugar}.
\begin{equation} \label{eq:grad_plato_to_density}
 d_{total} = 4.13 \frac{g}{l \,\,\, ^\circ P} p + 997 \frac{g}{l}
\end{equation}

As the simulation framework expects metabolite densities relative to the total volume of the solution (mmol of metabolite per liter
solution, mmol/l) the total density $d_{total}$ must to converted to a density $s_{glc}$. It is assumed that $V_{total} = V_{glc} + V_W$.

\begin{equation*}
 d_{total} = \frac{m_{total}}{V_{total}}
\end{equation*}
\begin{equation*}
 d_{total} = \frac{m_{glc} + m_w}{V_{total}}
\end{equation*}
\begin{equation*}
 d_{total} = \frac{m_{glc} + d_w V_w}{V_{total}}
\end{equation*}
\begin{equation*}
 d_{total} = \frac{m_{glc} + d_w \left(V_{total} - V_{glc}\right)}{V_{total}}
\end{equation*}
\begin{equation*}
 d_{total} = \frac{m_{glc} + d_w \left(V_{total} - \frac{m_{glc}}{d_{glc}}\right)}{V_{total}}
\end{equation*}
\begin{equation} \label{eq:total_density_to_metabolite_density}
 s_{glc} = \frac{m_{glc}}{V_{total}} = \frac{d_{total} - d_w}{1 - \frac{d_w}{d_{glc}}}
\end{equation}

Combining equation \ref{eq:grad_plato_to_density} and \ref{eq:total_density_to_metabolite_density}, including all constants and converting it to mmol/l leads to:
\begin{equation} \label{eq:ready_to_use_plato_to_metabolite_density}
 s_{glc} = \left( 63.857 \frac{1}{^\circ P} p - 46.385 \right) \frac{mmol}{l}
\end{equation}




\begin{table}[h]
\centering
\caption{Constants used in this document}
\label{tab:constants_used_in_this_document}
\begin{tabular}{llllll}
\rowcolor[HTML]{EFEFEF} 
\cellcolor[HTML]{EFEFEF} Constant    & \cellcolor[HTML]{EFEFEF}symbol      & \cellcolor[HTML]{EFEFEF}value & \cellcolor[HTML]{EFEFEF}reference\\
Oxygen saturation of water at 20\textdegree C (mg/l) & -  & 9.077 & \cite{fao_water_1987} \\
Molar mass of water (g/mol)    & -   & 18.015 & \cite{pupchen_website}\\
Molar mass of glucose (g/mol) & -  & 180.156 & \cite{pupchen_website}\\
Density of water (g/l) & $d_w$ & 1.00 &  \cite{pupchen_website}\\
Density of glucose (g/l) & $d_{glc}$ &  1.56 & \cite{pupchen_website}\\
Typical glucose/water solution density to brew beer (\textdegree P) & - & 5...20 & - \\
\end{tabular}
\end{table}

To calculate the initial oxygen density in the solution it is assumed that the solution is at 20 \textdegree C and fully saturated
with oxygen:
\begin{equation} \label{eq:init_oxygen_density}
s_{init,ox} = 9.077 \frac{mg}{l} = 9.077 \,\, 10^{-3}  \frac{g}{l} = \frac{9.077 \,\, 10^{-3} \frac{g}{l}}{18.015 \,\, 10^{-3} \frac{g}{mmol}} = 0,5039 \frac{mmol}{l} 
\end{equation}

% you can choose not to have a title for an appendix
% if you want by leaving the argument blank
\section{}\label{ap:more_super_fancy_stuff}
Appendix two text goes here.
