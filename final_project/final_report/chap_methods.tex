\section{Methods}\label{sec:methods}

Zomorri et al. summarizes in \cite{zomorrodi_synthetic_2016} models to predict the behavior of bacteria cultures and introduces
different categories. Three of them are especially
interesting to be used in this project: \textit{steady-state models}, \textit{spatio-temporal models} and \textit{dynamic models}.
\textit{Steady-state models} like compartmentalized community-level metabolic modeling can not be used
since a common objective can not be generally assumed, as it would be in a purely competitive co-cultures
\textit{Spatio-temporal models} have a very high computational effort as they take spacial and temporal varying bacteria densities
into account. As the spacial aspect is not necessarily required in this project a more optimal approach shall be preferred.
The remaining category of \textit{dynamic models} is a well established method to simulate microbial co-cultures in batch processes
and summarizes different extensions to dynamic flux balance analyses methods (DFBA)\cite{zomorrodi_synthetic_2016}. They use genome-
scale models (GEM) to simulate the behavior of the bacteria cultures and add differential equations to model the external system dynamics.

Mehadevan et al. introduces two basic categories of DFBA approaches: \textit{dynamic optimization approach} (DOA) and \textit{static
optimization approach} (SOA)\cite{mahadevan_dynamic_2002}. In DOA a the linear programming problem (LP) which predicts the bacteria
behavior is reformulized to a non-linear programming problem (NLP). This approach has a very high computational effort
\cite{hoffner_reliable_2013} compared to SOA and has only been used for relatively small GEMs with up to 13 modeled fluxes and 8 metabolites
\cite{luo_dynamic_2006} \cite{luo_photosynthetic_2009}.

Mehadevan et al. introduces SOA in \cite{mahadevan_dynamic_2002} as follows: The simulation interval is divided into several intervals and the LP is solved for each
of these time intervals dependent on the metabolite densities. The solution of the LP defines the bacteria growth and metabolite
production at a certain point of time in the simulation time interval. These values are then used to solve the differential equations
which models the external system dynamics. To solve the LP for the next time interval the new calculated, changed metabolite densities
are used. This procedure is repeated until the end of the simulation time interval is reached. This approach makes use of the
assumption that the cell internal dynamics are much faster than the external dynamics. In SOA the behavior of the bacteria is assumed
to be constant during one time interval what leads to a linear approximation approach when solving the system of ordinary differential
equations (ODE), similar to Euler-Cauchy methods.

Höffner et al. adds in \cite{hoffner_reliable_2013} a further group, the \textit{direct approach} (DA) which basically describes methods
similar to SOA which uses an ODE solver instead of the Euler-Cauchy method. Due to the used ODE solver different numerical approximation
methods can be used, not only the linear approximation. A good documented example for this group is the \textit{Dynamic Multispecies
Metabolic Modeling} framework by Zhuang et al. \cite{zhuang_design_2012}.

Henson et al. mentions a third group, \textit{reformulation to a differential-glgebraic equation system} \cite{henson_dynamic_2014}.
It shows also many similarities to SOA with the difference that the LP is reformulized but still solved as a LP embedded within 
the external ODE. The reformulated equation system makes it possible to enhance the efficiency of algorithm compared to SOA and DA
\cite{hoffner_reliable_2013}.

\begin{table}[]
\centering
\caption{Rating of considered DFBA methods}
\label{tab:rating_of_DFBA_methods}
\begin{tabular}{llll}
\rowcolor[HTML]{EFEFEF} 
Method                                                                                                 & \begin{tabular}[c]{@{}l@{}}comp.\\ effort\end{tabular} & \begin{tabular}[c]{@{}l@{}}impl.\\ complexity\end{tabular} & flexibility \\
dynamic optimization approach (DOA)                                                                    & high                                                   & medium-high                                                & ?           \\
static optimization approach (SOA)                                                                     & low                                                    & low                                                        & low         \\
direct approach (DA)                                                                                   & medium                                                 & low                                                        & medium      \\
\begin{tabular}[c]{@{}l@{}}reformulation to a differential-glgebraic\\ equation system\end{tabular} & low-medium                                             & high                                                       & ?          
\end{tabular}
\end{table}

The described DFBA methods in section \ref{ssec:considered_dfba_approaches} were rated based on the given information in the above mentioned
papers, see table \ref{tab:rating_of_DFBA_methods}.

DOA can not be used due to its high computational effort and medium-high implementation complexity. The approach which uses
\textit{reformulation to a differential-glgebraic equation system} is currently available in matlab code and must be implemented
in python in this project. Due to the high implementation complexity this approach will also be excluded.
The remaining methods, SOA and DA, have similar ratings but as DA is more flexible as different ODE solvers can be used this approach
seems more sustainable. Besides its flexibility the DA implementation DMMM by Zhuang et al. \cite{zhuang_design_2012} can be publicly
accessed and they provide a good documentation which will facilitate the implementation in this project.











\begin{table}[h]
\centering
\caption{Model constants used in the simulation setup}
\label{tab:model_constants_simulation_setup}
\begin{tabular}{llllll}
\rowcolor[HTML]{EFEFEF} 
\cellcolor[HTML]{EFEFEF} Constant          & \cellcolor[HTML]{EFEFEF}S. cerevisiae & \cellcolor[HTML]{EFEFEF}L. plantarum\\
Maximum glucose uptake rate (mmol/g/h)     & 18.5 & ? \\
Maximum oxygen uptake rate (mmol/g/h)      & 2.5 & ? \\
Glucose uptake saturation constant (g/l)   & 0.5 & ? \\
Oxygen uptake saturation constant (mM)     & 0.005 & ? \\
\end{tabular}
\end{table}

\begin{table}[h]
\centering
\caption{Simulation parameters used in the simulation setup}
\label{tab:simulation_parameters_simulation_setup}
\begin{tabular}{llllll}
\rowcolor[HTML]{EFEFEF} 
\cellcolor[HTML]{EFEFEF} Parameter          & \cellcolor[HTML]{EFEFEF}value & \cellcolor[HTML]{EFEFEF}reference\\
Initial glucose density (mmol/l) & 272.9 ... 1230.755 & equation \ref{eq:ready_to_use_plato_to_metabolite_density}, table \ref{tab:constants_used_in_this_document} \\
Initial oxygen density (mmol/l)  & 0.5039 & equation \ref{eq:init_oxygen_density}, table \ref{tab:constants_used_in_this_document}\\
\end{tabular}
\end{table}

\begin{table}[h]
\centering
\caption{Overview of implemented features compared to DMMM}
\label{tab:overview_implemented_features_compared_to_dmmm}
\begin{tabular}{llllll}
\rowcolor[HTML]{EFEFEF} 
\cellcolor[HTML]{EFEFEF} Feature                  & \cellcolor[HTML]{EFEFEF}DMMM & \cellcolor[HTML]{EFEFEF}This project\\
Model                                    &   &  \\
\hspace{0.5cm}arbitrary many GEMs & yes & yes \\
\hspace{0.5cm}\begin{tabular}[c]{@{}l@{}}arbitrary many metabolites\\ in environment\end{tabular} & yes & yes \\
\hspace{0.5cm}mortablility of bacteria & \begin{tabular}[c]{@{}l@{}}yes\\(in output flux)\end{tabular} & yes \\
\hspace{0.5cm}\begin{tabular}[c]{@{}l@{}}input/output flux of bacteria\\ and metabolites\end{tabular} & yes & no \\
\hspace{0.5cm}\begin{tabular}[c]{@{}l@{}}parameterized initial state\\ of environment composition\end{tabular} & yes & yes \\
\hspace{0.5cm}Michaelis-Menten kinetics & yes & yes \\
Algorithm &  &  \\
\hspace{0.5cm}ODE solver & yes & yes \\
\hspace{0.5cm}different ODE solvers & yes & no \\
\hspace{0.5cm}analytical solver & yes &no \\
\end{tabular}
\end{table}

As described by Zhuang et al. in \cite{zhuang_genome-scale_2011} the algorithm uses a ODE solver with embedded FBA. A FBA is solved
for each GEM in the model and for each time step in the discretised simulation time interval considering the changed metabolite and
bacteria densities in the shared environment. The results of the FBAs are used by the ODE solver to solve the differential equations

\begin{equation} \label{eq:diff_eq_x}
 \frac{\mathrm d x_j}{\mathrm d t} = \mu_j x_j
\end{equation}
\begin{equation} \label{eq:diff_eq_s}
 \frac{\mathrm d s_i}{\mathrm d t} = \displaystyle\sum_{j=1}^{N} v_{i,j} x_j
\end{equation}

which models the dynamics of the bacteria's environment \cite{zhuang_design_2012} where $i = 1...N$ is the index of metabolites in the shared environment and $j = 1...M$ is the index of bacteria.
The bacteria density is modeled in $x_j$ with $\left[ x_j \right] = \frac{g}{l}$ and $\mu_j$ is the bacteria's growth rate with $\left[ \mu_j \right] = \frac{mmol}{g_{DW} h}$.
Input and output fluxes of the bacteria's models are modeled in $v_{i,j}$ with $\left[ v_{i,j} \right] = \frac{mmol}{g_{DW} h}$,
the densities of metabolites in the shared environment in $s_i$ with $\left[ s_i \right] = \frac{mmol}{l}$.

In each time step each bacteria's metabolite intake must be changed dependent on the densities of the metabolites in the shared environment.
To model saturation of metabolite intake for high metabolite densities Zhuang et al. implemented Michaelis-Menten kinetics \cite{johnson2011original}

\begin{equation} \label{eq:michaelis-menten}
 v_{max,i,j} = \frac{v_{mm,i,j} s_i}{s_i + k_{mm,i,j}}
\end{equation}

This formula describes the upper bound of the input flux $v_{max,i,j}$ for metabolite i of bacteria j dependent on the metabolite density
$s_i$. The formula is characterized by to constants $\left[ v_{mm,i,j} \right] = \frac{mmol}{g_{DW} h}$ and $\left[ k_{mm,i,j} \right] = \frac{mmol}{l}$
for each bacteria and metabolite.

Mortality is considered using a constant $\left[ \mu_{mort,j} \right] = \frac{mmol}{g_{DW} h}$ for each bacteria j in this implementation while Zhuang et al. modeled this
using the output flux of bacteria out of the system.

Algorithm \ref{alg:differential_equation_with_embedded_fba} shows a basic implementation of the differential equations solved by an ODE
solver during the simulation similar to DMMM \cite{zhuang_genome-scale_2011}.

The algorithm expects a list of bacteria models consisting of
\begin{itemize}
 \item GEM of this bacteria: A, $\bm{v_{min}}$, $\bm{v_{max}}$, $\bm{w_{growth}}$
 \item $\bm{v_{mm}}$ (Michaelis-Menten $V_{max}$) for each exchange metabolite and species
 \item $\bm{k_{mm}}$ (Michaelis-Menten K) for each exchange metabolite and species
 \item mortality $\mu_{mort}$
\end{itemize}

Furthermore a list of all exchange metabolites in the environment, the bacteria and metabolite densities.


\begin{algorithm}
    \SetKwInOut{Input}{Input}
    \SetKwInOut{Output}{Output}

    \underline{function step}$(model_1...model_M, m_1...m_N, x_1...M, s_1...s_N)$\;
    \Input{bacteria models $model_j$, exchange metabolites $m_i$ in environment, bacteria densities $x_j$, metabolite densities $s_i$}
    \Output{slope of bacteria and metabolite densities $\dot{x}_j, \dot{s}_i$}
    \For{$j := 1$ \KwTo M}{
      \For{$i := 1$ \KwTo M}{
%	$\bm{v_{max,j}}[m] := michaelis\_menten(\bm{vmm_{j}}[m], \bm{kmm_{j}}[m], s[m])$
	$model_j := update\_intake\_bounds(model_j, s_j, m_i)$
      }
    }
    \For{$j := 1$ \KwTo M}{
      $\mu_j, \bm{v_j} := FBA(model_j, \bm{w_{growth}})$
    }
    $\bm{\mu} := \bm{\mu} - \bm{\mu_{mort}}$\\
    $\dot{\bm{x}} := diag(\bm{\mu}) \, \bm{x}$\\
    \For{$j := 1$ \KwTo M}{
      \For{$i := 1$ \KwTo N}{
        $\bm{\dot{s}}[m_i] := \bm{\dot{s}}[m_i] + \bm{v_j}[m_i] x_j$
      }
    }
    return $\dot{\bm{x}}$, $\dot{\bm{s}}$
    \caption{Differential equation with embedded FBA}
    \label{alg:differential_equation_with_embedded_fba}
\end{algorithm}

In a first step the upper bounds of the intake fluxes are updated for each bacteria j and exchange metabolite i.
The function $update\_intake\_bounds(model_j, s_j, m_i)$ calculates the upper bounds using the formula \ref{eq:michaelis-menten} if
the metabolite $m_i$ is contained in $model_j$ as a exchange metabolite and updates this value in the model.

In a next step the GEMs are optimized for growth using FBA, the results are used as growth rate $\mu_j$ and actual input and output
fluxes $\bm{v_j}$ of bacteria j in this time step.

The mortality is considered by subtracting the constants $\bm{\mu}$ from the growth rates $\bm{\mu}$.

At last step the slopes $\dot{\bm{x}}$ and $\bm{\dot{s}}$ are calculated according to \ref{eq:diff_eq_x} and \ref{eq:diff_eq_s} and
returned to the ODE solver.