\section{Methods}\label{sec:methods}

\subsection{Genome-Scale Models}\label{ssec:genome_scale_models}

\subsection{Flux Balance Analysis}\label{ssec:flux_balance_analysis}

\subsection{Simulation Algorithm}

\begin{table}[h]
\centering
\caption{Overview of implemented features compared to DMMM}
\label{tab:overview_implemented_features_compared_to_dmmm}
\begin{tabular}{llllll}
\rowcolor[HTML]{EFEFEF} 
\cellcolor[HTML]{EFEFEF} Feature                  & \cellcolor[HTML]{EFEFEF}DMMM & \cellcolor[HTML]{EFEFEF}This project\\
Model                                    &   &  \\
\hspace{0.5cm}arbitrary many GEMs & yes & yes \\
\hspace{0.5cm}\begin{tabular}[c]{@{}l@{}}arbitrary many metabolites\\ in environment\end{tabular} & yes & yes \\
\hspace{0.5cm}mortablility of bacteria & \begin{tabular}[c]{@{}l@{}}yes\\(in output flux)\end{tabular} & yes \\
\hspace{0.5cm}\begin{tabular}[c]{@{}l@{}}input/output flux of bacteria\\ and metabolites\end{tabular} & yes & no \\
\hspace{0.5cm}\begin{tabular}[c]{@{}l@{}}parameterized initial state\\ of environment composition\end{tabular} & yes & yes \\
\hspace{0.5cm}Michaelis-Menten kinetics & yes & yes \\
Algorithm &  &  \\
\hspace{0.5cm}ODE solver & yes & yes \\
\hspace{0.5cm}different ODE solvers & yes & no \\
\hspace{0.5cm}analytical solver & yes &no \\
\end{tabular}
\end{table}

As described by Zhuang et al. in \cite{zhuang_genome-scale_2011} the algorithm uses a ODE solver with embedded FBA. A FBA is solved
for each GEM in the model and for each time step in the discretised simulation time interval considering the changed metabolite and
bacteria densities in the shared environment. The results of the FBAs are used by the ODE solver to solve the differential equations

\begin{equation} \label{eq:diff_eq_x}
 \frac{\mathrm d x_j}{\mathrm d t} = \mu_j x_j
\end{equation}
\begin{equation} \label{eq:diff_eq_s}
 \frac{\mathrm d s_i}{\mathrm d t} = \displaystyle\sum_{j=1}^{N} v_{i,j} x_j
\end{equation}

which models the dynamics of the bacteria's environment \cite{zhuang_design_2012} where $i = 1...N$ is the index of metabolites in the shared environment and $j = 1...M$ is the index of bacteria.
The bacteria density is modeled in $x_j$ with $\left[ x_j \right] = \frac{g}{l}$ and $\mu_j$ is the bacteria's growth rate with $\left[ \mu_j \right] = \frac{mmol}{g_{DW} h}$.
Input and output fluxes of the bacteria's models are modeled in $v_{i,j}$ with $\left[ v_{i,j} \right] = \frac{mmol}{g_{DW} h}$,
the densities of metabolites in the shared environment in $s_i$ with $\left[ s_i \right] = \frac{mmol}{l}$.

In each time step each bacteria's metabolite intake must be changed dependent on the densities of the metabolites in the shared environment.
To model saturation of metabolite intake for high metabolite densities Zhuang et al. implemented Michaelis-Menten kinetics \cite{johnson2011original}

\begin{equation} \label{eq:michaelis-menten}
 v_{max,i,j} = \frac{v_{mm,i,j} s_i}{s_i + k_{mm,i,j}}
\end{equation}

This formula describes the upper bound of the input flux $v_{max,i,j}$ for metabolite i of bacteria j dependent on the metabolite density
$s_i$. The formula is characterized by to constants $\left[ v_{mm,i,j} \right] = \frac{mmol}{g_{DW} h}$ and $\left[ k_{mm,i,j} \right] = \frac{mmol}{l}$
for each bacteria and metabolite.

Mortality is considered using a constant $\left[ \mu_{mort,j} \right] = \frac{mmol}{g_{DW} h}$ for each bacteria j in this implementation while Zhuang et al. modeled this
using the output flux of bacteria out of the system.

Algorithm \ref{alg:differential_equation_with_embedded_fba} shows a basic implementation of the differential equations solved by an ODE
solver during the simulation similar to DMMM \cite{zhuang_genome-scale_2011}.

The algorithm expects a list of bacteria models consisting of
\begin{itemize}
 \item GEM of this bacteria: A, $\bm{v_{min}}$, $\bm{v_{max}}$, $\bm{w_{growth}}$
 \item $\bm{v_{mm}}$ (Michaelis-Menten $V_{max}$) for each exchange metabolite and species
 \item $\bm{k_{mm}}$ (Michaelis-Menten K) for each exchange metabolite and species
 \item mortality $\mu_{mort}$
\end{itemize}

Furthermore a list of all exchange metabolites in the environment, the bacteria and metabolite densities.


\begin{algorithm}
    \SetKwInOut{Input}{Input}
    \SetKwInOut{Output}{Output}

    \underline{function step}$(model_1...model_M, m_1...m_N, x_1...M, s_1...s_N)$\;
    \Input{bacteria models $model_j$, exchange metabolites $m_i$ in environment, bacteria densities $x_j$, metabolite densities $s_i$}
    \Output{slope of bacteria and metabolite densities $\dot{x}_j, \dot{s}_i$}
    \For{$j := 1$ \KwTo M}{
      \For{$i := 1$ \KwTo M}{
%	$\bm{v_{max,j}}[m] := michaelis\_menten(\bm{vmm_{j}}[m], \bm{kmm_{j}}[m], s[m])$
	$model_j := update\_intake\_bounds(model_j, s_j, m_i)$
      }
    }
    \For{$j := 1$ \KwTo M}{
      $\mu_j, \bm{v_j} := FBA(model_j, \bm{w_{growth}})$
    }
    $\bm{\mu} := \bm{\mu} - \bm{\mu_{mort}}$\\
    $\dot{\bm{x}} := diag(\bm{\mu}) \, \bm{x}$\\
    \For{$j := 1$ \KwTo M}{
      \For{$i := 1$ \KwTo N}{
        $\bm{\dot{s}}[m_i] := \bm{\dot{s}}[m_i] + \bm{v_j}[m_i] x_j$
      }
    }
    return $\dot{\bm{x}}$, $\dot{\bm{s}}$
    \caption{Differential equation with embedded FBA}
    \label{alg:differential_equation_with_embedded_fba}
\end{algorithm}

In a first step the upper bounds of the intake fluxes are updated for each bacteria j and exchange metabolite i.
The function $update\_intake\_bounds(model_j, s_j, m_i)$ calculates the upper bounds using the formula \ref{eq:michaelis-menten} if
the metabolite $m_i$ is contained in $model_j$ as a exchange metabolite and updates this value in the model.

In a next step the GEMs are optimized for growth using FBA, the results are used as growth rate $\mu_j$ and actual input and output
fluxes $\bm{v_j}$ of bacteria j in this time step.

The mortality is considered by subtracting the constants $\bm{\mu}$ from the growth rates $\bm{\mu}$.

At last step the slopes $\dot{\bm{x}}$ and $\bm{\dot{s}}$ are calculated according to \ref{eq:diff_eq_x} and \ref{eq:diff_eq_s} and
returned to the ODE solver.


\subsection{Simulation Setup}\label{ssec:simulation_setup}

The goal of the simulation is to validate the basic functionality of the simulator using a simplified setup of a realistic future
simulation scenario. As defined in our project goals, this simulation scenario is the dynamic flux balance analysis (DFBA) of a
co-culture of Saccharomyces cerevisiae and Lactobacillus plantarum.

As genome-scale models a model of Lactobacillus plantarum published by Teusink et al. \cite{teusink_analysis_2006}. A decision about
a yeast model is not made yet.

The simulation will consider two input metabolites: oxygen and glucose. Table \ref{tab:model_constants_simulation_setup} and table
\ref{tab:simulation_parameters_simulation_setup} contain all values needed to define the initial metabolite conditions and kinetics.

\begin{table}[h]
\centering
\caption{Model constants used in the simulation setup}
\label{tab:model_constants_simulation_setup}
\begin{tabular}{llllll}
\rowcolor[HTML]{EFEFEF} 
\cellcolor[HTML]{EFEFEF} Constant          & \cellcolor[HTML]{EFEFEF}S. cerevisiae & \cellcolor[HTML]{EFEFEF}L. plantarum\\
Maximum glucose uptake rate (mmol/g/h)     & 18.5 & 18.5 \\
Maximum oxygen uptake rate (mmol/g/h)      & 2.5 & 2.5 \\
Glucose uptake saturation constant (g/l)   & 0.5 & 0.5 \\
Oxygen uptake saturation constant (mM)     & 0.005 & 0.005 \\
Mortability (?)                            & ? & ? \\
\end{tabular}
\end{table}

% E-Mail from Steinn about model constants of L. plantarum from 12.04.2018:
% ----------------------------------------------------------------------------------------------
% As a first approximation I would assume the same v_max and k_m for L. plantarum as in yeast.
% 
% The maximum glucose uptake rate can be assumed to be similar to yeast for the time being 
% (for Lactobacillus reuteri it is approx. 21 mmol/gDW/h [gDW = grams dry weight]). L. plantarum 
% is an anaerobe that can tolerate oxygen (and has some oxygen dependent metabolism as well) but
% as a first approximation I would assume that oxygen plays a minor role, hence the oxygen
% saturation constant may not be too important.
% ----------------------------------------------------------------------------------------------

\begin{table}[h]
\centering
\caption{Simulation parameters used in the simulation setup}
\label{tab:simulation_parameters_simulation_setup}
\begin{tabular}{llllll}
\rowcolor[HTML]{EFEFEF} 
\cellcolor[HTML]{EFEFEF} Parameter          & \cellcolor[HTML]{EFEFEF}value & \cellcolor[HTML]{EFEFEF}reference\\
Initial glucose density (mmol/l) & 272.9 ... 1230.755 & equation \ref{eq:ready_to_use_plato_to_metabolite_density}, table \ref{tab:constants_used_in_this_document} \\
Initial oxygen density (mmol/l)  & 0.5039 & equation \ref{eq:init_oxygen_density}, table \ref{tab:constants_used_in_this_document}\\
Initial density of S. cerevisiae (mmol/l) & ? & - \\
Initial density of L. plantarum (mmol/l) & ? & - \\
\end{tabular}
\end{table}

To verify the basic functionality of the simulator the resulting bacteria densities and metabolite densities of ethanol, d- and l-lactate,
oxygen and glucose will be compared to existing data.
