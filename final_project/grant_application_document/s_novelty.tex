% section: novelty

\noindent
\textbf{Describe the novelty of the project and how the results from it can be applied to the benefit of society}

Until now GEMs has not been used to investigate the fermentation process of wort, so this project has the capability to provide completely
new insights into this process. As beer brewing is applied all around the world and in huge amounts, results in this field can have a big
impact on production processes.
Additionally, the simulation framework will use only free available and open-source software while the most existing implementations are written in matlab
which is a commercial product and the source code can not be viewed.

%Novelty:
%\begin{itemize}
% \item GEMs have not been used in the field of beer brewing
% \item other implementations of algorithms are mostly in matlab and use costly modules
% \item our approach will be open-source and available for free
%\end{itemize}

As this implementation can be used completely free and is open source it is very easy accessible and can be used in projects with low budgeds
like in smaller companies, start-ups, institutes and academic research and can also be used by students to learn more about this topic.
The easy accessibility will facilitate the development in this field and helps to create new solutions.

%Application of results:
%\begin{itemize}
% \item The simulation framework:
% \begin{itemize}
 % \item is completely free available and open source
 % \item can also be used by students to learn more about this topic
 % \item can be used by smaller companies (cheaper than matlab)
 % \item can be interesting by academic research (cheaper than matlab)
 % \item will enable new development in the field of co-culture simulation
 % \item will help to create new solutions how to generate metabolites in a co-culture of several bacterias
 %\end{itemize}

% \item The simulation results:
% \begin{itemize}
%  \item The simulation environment and the setup will be available for free
%  \item can be used by smaller breweries to
%  \begin{itemize}
%   \item the quality of their beer
%   \item reducing costs of production: less cleaning effort, more efficient processes,...
%  \end{itemize}
%  \item reduces the impact on the environment: less usage of chemicals for cleaning
% \end{itemize}
%\end{itemize}

New insight in the fermentation process of wort will lower the production costs, increase the durability and enhance the quality of the
produced beer. Furthermore it has the potential to reduces the need for high aggressive chemicals to clean the production tanks which
lowers the environmental impact.


%\noindent
%\textbf{Describe how the project can lead to the development of goods, processes or services}
%\begin{itemize}
% \item some natural co-cultures are not culturable due to dependencies \cite{d2010siderophores}
% \item can help to find solutions how to treat human diseases \cite{ZOMORRODI2016837}
% \item reduce uranium concentration in contaminated areas \cite{zhuang2011genome}
% \item biofuel \cite{hanly2011dynamic} \cite{chiu2014emergent}
%\end{itemize}

Interspecies communication and dependencies in microbial co-cultures are widely unexplored and some natural co-cultures are still not
culturable due to missing knowledge about their dependencies \cite{d2010siderophores}. Online services provide GEMs to a variety of bacteria.
This simulation framework can predict the growth and substrate dynamics of such cultures based on these GEMs. The results can be compared to
lab experiments and used to refine the interspecies models iteratively. Also other highly sustainable areas can be addressed with this
framework like the development of treatments of human diseases \cite{ZOMORRODI2016837}, the reduction of uranium concentrations in
conterminated areas \cite{zhuang2011genome} or to produce biofuel \cite{hanly2011dynamic} \cite{chiu2014emergent}.

\noindent
\textbf{Which technological and design challenges are posed by the project}

\begin{itemize}
 \item Framework:
 \begin{itemize}
  \item the most implementations are in matlab, ours will be in python
  \item python does not provide many libraries as they are available in matlab
  \item it will be neccessary to find alternative ways for implementation / adapt existing algorithms to available ressorces in python
  \item to gurantee the stability of the algorithm (cite paper)
  \item find a compromise between a simple implementation which is still flexible and modular enough that it can be easily extended to refine the simulation setup/environment
  \item dynamic simulation using GEMs is still under research
  \item there are still many open problems
  \item ...maybe you find some disadvantages you could mention here. Maybe some of them could be a implementation goal to solve them...?
 \end{itemize}
 \item Simulation (setup):
 \begin{itemize}
  \item a realistic simulation of the conterminated fermentation needs a lot of knowledge about real medium conditions
  \item conditions must most probably be measured somehow
  \item conditions vary from setup to setup, so a stochastic model of the starting conditions must be created
  \item many modelling constants in the dynamic model must be measured as they are not available in publications (e.g. gas saturation, metabolite input ratios / saturation,...)
  \item temperatur dependency is often not modeled in GEMs
  \item GEMs are not accurate enough to generate realistic data from scratch (results must be compared with real cultures)
 \end{itemize}
\end{itemize}