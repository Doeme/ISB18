% section: description

\noindent
\textbf{Describe the project and its objectives}

%What is our goal?
%\begin{itemize}
% \item Create a simulation framework and use fermentation as proof of concept?
% \item Goal here:
% \begin{itemize}
%  \item Implement a simulation framework
%  \item The simulation is just a by-product but can be enhanced
% \end{itemize}
% \item Advantages:
% \begin{itemize}
%  \item a realistic simulation of the conterminated fermentation needs a lot of knowledge about real medium conditions
%  \item conditions must most probably be measured somehow
%  \item conditions vary from setup to setup, so a stochastic model of the starting conditions must be created
%  \item many modelling constants in the dynamic model must be measured as they are n
 % \item define values to quantify the quality of the product after fermentation
%  \item investigate correlations between starting conditions and the quality of the productot available in publications (e.g. gas saturation, metabolite input ratios / saturation,...)
%  \item temperatur dependency is often not modeled in GEMs
%  \item GEMs are not accurate enough to generate realistic data from scratch (results must be compared with real cultures)
%  \item it is very unrealistic to get to real results in this work if a fermentation simulation shall be the goal
%  \item if the simulation framework is the goal, it is much easier to argue why a simplified model was used
%  \item the simulation can still be enhanced to a more realistic setup
% \end{itemize}
% \item Simulate the fermentation and implement a simplified model as proof of concept?
%\end{itemize}

Traditional beer brewing is done in a batch process by fermenting glucose to alcohol using Saccharomyces cerevisiae. In a first step wort
is produced by mixing water, malt and hop and applying different enzymatical processes. As in a batch process
the densities of metabolites in the culture are not controlled the fermentation product and so the quality of the beer is highly dependent
on the composition of the wort. To enhance the product quality the process of fermentation must be understood in detail and correlations
between the starting conditions and fermentation results must be found. This is typically done in experiments which are very time and cost
intensive. Especially high effort is needed to reproduce starting conditions and if different yeast mutants or contamination by other 
bacteria shall be tested. These experiments are very costly and are not affordable for smaller breweries. A simulation approach to test
different starting conditions will reduce the amount of experiments and so the costs and will enable development of new production methods
also for smaller companies.
To be able to simulate the fermentation outcome three major requirements must be fulfilled: (1) The starting conditions and dynamic environment
constants must be determined, (2) the involved bacteria and internal (enzymatical) processes must be modeled sufficiently and (3) a proper
simulation framework must be available which combines all information of (1) and (2) and calculates the fermentation outcome. As the 
formulation of sufficient models for (1) and (2) depends on the production process, so the applied yest and wort, this project will
concentrate on the development of a simulation framework to enable the simulation of the fermentation products dependent on the fermentation's
starting conditions.
To be usable to enhance the fermentation process the simulation framework will enable the access to different process and product parameters.
The growth and metabolite production of Saccharomyces cerevisiae in a typical environment including contamination with other bacteria in a
batch process will be simulated. The simulation will be parameterised by the used bacteria models and environmental constants which defines
the composition and dynamics of the wort (starting conditions). The simulation results are used to formulate quality measures of the fermentation
product to finally investigate correlations to the simulation's starting conditions.
In order to find an optimal solution for the design and implementation of the simulation framework the project is divided into several steps.
In a first step detailed requirements to the simulation framework will be elaborated based on the goals described above. In a second step
a software design and simulation algorithm will be chosen which fulfill the previously defined requirements. The third step consists of the
implementation of the software what will be done in a software development typical iterative manner. The iterations include implementation,
test and refinement until all requirements are met. In a final step the framework will be applied to a simplified fermentation process setup
and will serve as a proof of concept. The simplified setup contains only Lactobacillus Brevisas as contamination bacteria. As a measure of
the product quality the density of ethanol and lactic acids shall be used. The contamination with Lactobacillus Brevisas is a typical
scenario which can lead to the effect that the beer turns sour under certain conditions \cite{JIB:JIB49}.
If the project progress allows further development, additional design goals can be implemented like an generic or automated integration of
bacteria models, an optimized user interface, a graphical user interface or an enhanced model of the fermentation process.

%Strategy:
%\begin{enumerate}
% \item What is the problem?
% \begin{itemize}
%  \item In many breweries the beer is fermented in a batch culture
%  \item the quality of the product is highly dependent on the starting conditions before the fermentation starts
% \end{itemize}
% \item What can solve the Problem?
% \begin{itemize}
%  \item Ideal starting conditions must be found
%  \item experimental approach is very time and cost intensive
%  \item simulation of fermentation is better
% \end{itemize}
% \item What must be simulated?
% \begin{itemize}
%  \item growth and metabolite production of Saccharomyces cerevisiae in a beer-brewery typical environment shall be analysed
%  \item the yeast shall be simulated together with typical contamination bacteria to create realistic fermentation conditions
%  \item the simulation of the fermentation shall be dependent of starting conditions
 % \item define values to quantify the quality of the product after fermentation
%  \item investigate correlations between starting conditions and the quality of the product
% \end{itemize}
% \item What is the goal of this project?
% \begin{enumerate}
%  \item define requirements to framework
%  \item view publications about possible algorithms and choose one
%  \item implement chosen algorithm
%  \item Proof of concept:
%  \begin{itemize}
%   \item proof the concept of the implemented simulation framework with a simplified setup of the described experiment
%   \item the simplified setup contains only Lactobacillus Brevisas as contamination bacteria
%   \item the density of ethanol and lactic acids shall be used to rate the quality of the product
%   \item this simplified setup shall simulate the most common contamination and the effect that the beer turns sour under certain starting conditions
%   \item \cite{JIB:JIB49}
%  \end{itemize}
%  \item Nice to have
%  \begin{itemize}
%   \item generic/automatic GEM integration
%   \item nice UI/GUI
%   \item enhancing the simplified model of the contaminated fermentation to a more realistic one
%  \end{itemize}
% \end{enumerate}
%\end{enumerate}

\noindent
\textbf{Describe the state of the art in the field}
\begin{itemize}
 \item short overview over available techniques
 \item discuss advantages and disadvantages of available techniques
\end{itemize}

There have been many approaches to simulate microbial communities \cite{ZOMORRODI2016837}. To model the behavior of the simulated bacteria
genome-scale models (GEM) are used as their usability in similar applications has been often proven and a variety of models are publicly
available \cite{6915830}.
Zomorrodi et al. presents in \cite{ZOMORRODI2016837} a comprehensive overview over different simulation approaches of microbial communities.
The methods ranges from steady-state models to dynamic models and spatio-temporal models. They all have in common that they use GEMs and
flux balance analysis (FBA) to evaluate the behavior of the involved bacteria for certain environmental conditions but they differ in the
variety of result aspects and so also in their computational effort.
Steady-state models like compartmentalized community-level metabolic modeling uses a multispecies stoichiometric metabolic model
\cite{Stolyar92} which is optimized using a mutual objective function. Other approaches as in OptCom use nested optimizations where again
a mutual objective function is used to connect the bacteria's models to each other \cite{zomorrodi2012optcom}. The principle to use a mutual
objective function has the disadvantage that these models can only by applied to microbial communities which has a common objective and does
not purely compete with each other \cite{ZOMORRODI2016837}. As such a behavior can not be excluded in fermentation processes with bacteria
contamination this approach can not be used in this project.
Another requirement is that the simulation algorithm must be able to simulate batch cultures. Steady-state models are only partly usable in
such simulations. A better option are here spatio-temporal and dynamic models. Both are able to take the varying densities of metabolites
into account. Spatio-temporal models additionally uses spatial distributions of the bacteria to predict their behavior. As such simulations
as a very high computational complexity this method must also be excluded in practical approaches as in this project \cite{ZOMORRODI2016837}.
Dynamic models use differential equations to predict the temporal behavior of the environment where the behavior of the bacteria is modeled
using GEMs and FBA.





\noindent
\textbf{Describe how the project can advance the state of the art in the field}

\begin{itemize}
 \item The simulation framework:
 \begin{itemize}
  \item simulation of co-culture is interesting for many areas
  \item (list a few examples)
  \item this framework is not dedicated to only one setup
  \item builds a basis for future simulation projects
  \item python has a bit programmers community which makes it easy to customize or enhance the framework
  \item framework can be used to implement new approaches
 \end{itemize}
 \item The simulation results:
 \begin{itemize}
  \item new and detailed knowledge about fermentation setup
  \item better control of fermentation process and product quality
  \item new solutions to ensure product quality
  \item estimation of product quality (max/min ratio of bad fermentation products)
  \item reduction of chemicals for cleaning
 \end{itemize}
\end{itemize}


